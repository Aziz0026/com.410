\documentclass[12pt,a4paper,oneside]{article}

\usepackage[margin=3cm]{geometry}

\usepackage{hyperref}
\hypersetup{
    pdftitle={COM 410, Computer Architecture},%
    pdfauthor={Toksaitov Dmitrii Alexandrovich},%
    pdfsubject={Syllabus},%
    pdfkeywords={COM;}{410;}{syllabus;}{computer;}{architecture},%
    colorlinks,%
    linkcolor=black,%
    citecolor=black,%
    filecolor=black,%
    urlcolor=black
}

\newcommand{\R}[1]{\uppercase\expandafter{\romannumeral #1\relax}}

\begin{document}

    \title{COM 410, Computer Architecture}
    \author{
        American University of Central Asia\\
        Software Engineering Program
    }
    \date{}
    \maketitle

    \section{Course Information}

        \begin{description}
            \item[Course ID]\hfill\\
                COM 410, 3268
            \item[Course Repository]\hfill\\
                \url{https://github.com/auca/com.410}
            \item[Class Discussions]\hfill\\
                \url{https://piazza.com/auca.kg/spring2019/com410}
            \item[Place]\hfill\\
                AUCA, room 410\\
                AUCA, laboratory G31
            \item[Time]\hfill\\
                Lecture: Monday 10:50--12:05\\
                Lab: Wednesday 10:50--12:05\\
                Lab: Wednesday 12:45--14:00\\
                Lab: Friday 10:50--12:05
        \end{description}

    \section{Prerequisites}

        \begin{itemize}
            \item COM-117, Object-Oriented Programming
            \item or COM-223, Algorithms and Data Structures
            \item or COM-311, Circuit Engineering
        \end{itemize}

    \section{Contact Information}

        \begin{description}
            \item[Instructor]\hfill\\
                Toksaitov Dmitrii Alexandrovich\\
                \href{mailto:toksaitov_d@auca.kg}{toksaitov\_d@auca.kg}
            \item[Teacher Assistants]\hfill\\
                Umarbaev Bektur\\
                \href{mailto:umarbaev_b@auca.kg}{umarbaev\_b@auca.kg}\\
                Samuel Ramaley Furr\\
                \href{mailto:furr_s@auca.kg}{furr\_s@auca.kg}
            \item[Office]\hfill\\
                AUCA, room 315
            \item[Office Hours]\hfill\\
                Sat, Sun (remotely through Skype at toksaitov@hotmail.com\\
                Additional office hours are scheduled by TAs
        \end{description}

    \section{Course Overview}

        The course introduces students to the topic of computer architecture and
        organization. Students will focus on studying the structure and design
        of modern central processing units. During lab sessions, students will
        learn the basics of the x86 instruction set, the assembly language for
        the platform mentioned above, and the representation of high-level
        language structures in the low-level language.

    \section{Topics Covered}

        \begin{itemize}
            \item The modern computer architectures and organization
            \item The x86 and x86-64 assembly languages
            \item Representation of high-level language structures in low-level assembly languages
            \item Acceleration with SIMD instructions
            \item System emulation
        \end{itemize}

    \section{Examinations}

        \subsection{Lectures}

            Students will have to take a midterm and final examinations on
            topics discussed during lectures. Each exam is in the form of a quiz
            with a set of open and multiple choice questions.

        \subsection{Labs}

            Students will have a number of laboratory tasks to finish on their
            own. Students will have to defend their work to the instructor
            during separate midterm and final examination sessions.

    \section{Course Projects}

        Throughout the course, students will have to work on one significant
        project. The work will require to accelerate an image processing
        application by optimizing the hot path of a C program in x86-64
        assembly.

    \section{Course Materials, Recordings and Screencasts}

        Students will find all the course materials on GitHub. We hope that by
        working with GitHub, students will become familiar with the Git version
        control system and the popular (among developers) GitHub service. Though
        version control is not the focus of the course, some course tasks may
        have to be submitted through it on the GitHub Classroom service.

        Every class is screencasted online and recorded to YouTube for students’
        convenience. An ability to watch classes remotely MUST NOT be the reason
        not to attend the class. Active class participation is necessary to
        succeed in this course.

    \section{Reading}

        \begin{itemize}
            \item Computer Architecture: A Quantitative Approach, 5th Edition by David Patterson and John L. Hennessy (ISBN: 978-0123838728)
            \item Assembly Language for x86 Processors, 7th Edition by Kip R. Irvine
        \end{itemize}

    \section{Grading}

        \begin{itemize}
            \item Class participation (through Piazza) (5\%)
            \item Lab Midterm (20\%)
            \item Lab Final (25\%)
            \item Lecture Midterm (10\%)
            \item Lecture Final (15\%)
            \item Course projects (30\%)
        \end{itemize}

        \begin{itemize} \itemsep-10pt \parskip0pt \parsep0pt
            \item[--] 90\%--100\%: A\\
            \item[--] 80\%--89\%: A-\\
            \item[--] 70\%--79\%: B+\\
            \item[--] 65\%--69\%: B\\
            \item[--] 60\%--64\%: B-\\
            \item[--] 56\%--59\%: C+\\
            \item[--] 53\%--55\%: C\\
            \item[--] 50\%--52\%: C-\\
            \item[--] 46\%--49\%: D+\\
            \item[--] 43\%--45\%: D\\
            \item[--] 40\%--42\%: D-\\
            \item[--] Less than 39\%: F
        \end{itemize}

    \section{Rules}

        Students are required to follow the rules of conduct of the Software
        Engineering Department and the American University of Central Asia.

        Teamwork is NOT encouraged. The same blocks of code or similar
        structural pieces in separate works will be considered as academic
        dishonesty, and all parties will get zero for the task.

        Attendance is mandatory. More than three misses without reason will
        result in 5 points being deducted from the student. If a student has
        health/family/personal emergency, he must notify the instructor if
        possible (through e-mail), to increases the chances for the miss to be
        not counted.

        Active work during the class may be awarded up to 10 points at the
        instructor’s discretion.

        Poor student performance during a class can lead to up to 3 points
        deducted from his final grade.

        Late submissions will receive a penalty of 10 points for every day after
        the deadline.

\end{document}

