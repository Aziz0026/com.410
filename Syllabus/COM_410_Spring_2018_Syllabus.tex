\documentclass[12pt,a4paper,oneside]{article}

\usepackage[margin=3cm]{geometry}

\usepackage{hyperref}
\hypersetup{
    pdftitle={COM 410, Computer Architecture},%
    pdfauthor={Toksaitov Dmitrii Alexandrovich},%
    pdfsubject={Syllabus},%
    pdfkeywords={COM;}{410;}{syllabus;}{computer;}{architecture},%
    colorlinks,%
    linkcolor=black,%
    citecolor=black,%
    filecolor=black,%
    urlcolor=black
}

\newcommand{\R}[1]{\uppercase\expandafter{\romannumeral #1\relax}}

\begin{document}

    \title{COM 410, Computer Architecture}
    \author{
        American University of Central Asia\\
        Department of Software Engineering
    }
    \date{}
    \maketitle

    \section{Course Information}

        \begin{description}
            \item[Course ID]\hfill\\
                COM 410, 3268
            \item[Course Repository]\hfill\\
                \url{https://github.com/auca/com.410}
            \item[Class Discussions]\hfill\\
                \url{https://piazza.com/auca.kg/spring2018/com410}
            \item[Place]\hfill\\
                AUCA, room 434\\
                AUCA, laboratory G30, G31
            \item[Time]\hfill\\
                Lecture: Monday 10:50--12:05\\
                Lab: Tuesday 10:50--12:05\\
                Lab: Wednesday 12:45--14:00
        \end{description}

    \section{Prerequisites}

        \begin{itemize}
            \item COM-117, Object-Oriented Programming
            \item or COM-223, Algorithms and Data Structures
            \item or COM-311, Circuit Engineering
        \end{itemize}

    \section{Contact Information}

        \begin{description}
            \item[Instructor]\hfill\\
                Toksaitov Dmitrii Alexandrovich\\
                \href{mailto:toksaitov_d@auca.kg}{toksaitov\_d@auca.kg}
            \item[Office]\hfill\\
                AUCA, room 315
            \item[Office Hours]\hfill\\
                Tuesday 15:35--18:00\\
                Thursday 15:35--18:00
        \end{description}

    \section{Course Overview}

        The course introduces students to the topic of computer architecture and
        organization. Students will focus on studying the structure and design
        of modern central processing units. During lab sessions students will
        learn basics of the x86 instruction set, the assembly language for the
        aforementioned platform, and the representation of high-level language
        structures in the low-level language.

    \section{Topics Covered}

        \begin{itemize}
            \item The modern computer architectures and organization
            \item The x86 and x86-64 assembly languages
            \item Representation of high-level language structures in low-level
            assembly languages
            \item Acceleration with SIMD instructions
            \item System Emulation
        \end{itemize}

    \section{Examinations}

        \subsection{Lectures}

            Students will have to take midterm and final examinations on topics
            discussed during lectures. Each examination is in the form of a quiz
            with a set of open and multiple choice questions.

        \subsection{Labs}

            Students will have a number of laboratory tasks to finish on their
            own. Students will have to defend their work to the instructor
            during separate midterm and final examination sessions.

    \section{Course Projects}

        Throughout the course, students will have to work on two major projects.
        The first work will require to accelerate an image processing
        application by optimizing the hot path of a C program in x86-64
        assembly. The second project will require to study an old computer
        architecture in details to write a software emulator for it.

    \section{Reading}

        \begin{itemize}
            \item Computer Architecture: A Quantitative Approach, 5th Edition by
            David Patterson and John L. Hennessy (ISBN: 978-0123838728)
            \item Assembly Language for x86 Processors, 7th Edition by Kip R.
            Irvine
        \end{itemize}

    \section{Grading}

        \begin{itemize}
            \item Class participation (through Piazza) (5\%)
            \item Lab Midterm (7.5\%)
            \item Lab Final (10\%)
            \item Lecture Midterm (7.5\%)
            \item Lecture Final (10\%)
            \item Course projects (60\%)
        \end{itemize}

        \begin{itemize} \itemsep-10pt \parskip0pt \parsep0pt
            \item[--] 90\%--100\%: A\\
            \item[--] 80\%--89\%: A-\\
            \item[--] 70\%--79\%: B+\\
            \item[--] 65\%--69\%: B\\
            \item[--] 60\%--64\%: B-\\
            \item[--] 56\%--59\%: C+\\
            \item[--] 53\%--55\%: C\\
            \item[--] 50\%--52\%: C-\\
            \item[--] 46\%--49\%: D+\\
            \item[--] 43\%--45\%: D\\
            \item[--] 40\%--42\%: D-\\
            \item[--] Less than 39\%: F
        \end{itemize}

    \section{Rules}

        Students are required to follow the rules of conduct of the Software
        Engineering Department and American University of Central Asia.

        Team work is NOT encouraged. The same blocks of code or similar
        structural pieces in separate works will be considered as academic
        dishonesty and all parties will get zero for the task.

\end{document}

